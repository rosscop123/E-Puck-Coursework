\section{Implementation}
\label{sec:implementation}
This section will describe how each of the tasks mentioned in section \ref{sec:design} have been implemented, explaining the functions within the code, and any discrepancies between the design and final implementation.

The design was gradually implemented in stages, building up to the final completed solution to each task. To make the project easier to implement the demo project provided on the E-Puck\cite{epuck} website was used as a template.
\subsection{Main File}
Within the demo project one can find the main file. This main file starts by finding the position of the selection, a single hexadecimal digit. This selector then decides which function will be called. This becomes very useful to the groups project as then each behaviour needed to be implemented can be ran when a unique selector is selected. Table \ref{table:selectorOptions} clearly shows the possible choices to the user with the code being displayed in the code snippet \ref{code:mainSelector}.

\begin{table}
	\centering
	\begin{tabular}{|c | c|} 
	\hline
	Selector & Function \\ \hline
	0 & run\_breitenberg\_follower \\ \hline
	1 & finding\_light \\ \hline
	2 & avoid\_light \\ \hline
	3 & run\_breitenberg\_shocker \\ \hline
	4 & followHand \\ \hline
	\end{tabular}
	\caption{A table to show what function is ran for each selector.}
	\label{table:selectorOptions}
\end{table}

The functions shown in table \ref{table:selectorOptions} will each be discussed in detail throughout this section.

\begin{lstlisting}[caption={Processes for selecting which function to run},label=code:mainSelector, float,floatplacement=H]
	selector=getselector();
	if (selector==0) {
		run_breitenberg_follower();
	} else if (selector==1) {
		finding_light();
	} else if (selector==2) {
		avoid_light();
	} else if (selector==3) {
		run_breitenberg_shocker();
	} else if (selector==4) {
		followHand();
	} else{
	}
\end{lstlisting}

\subsection{Breitenburg Follower}
The `run\_breitenburg\_follower' function is a function that was taken from the demo project mentioned previously. The function can be found in the `runBreitenberg\_adv.c' file along with it's dependencies.

The function uses the front 4 infrared sensors to attempt to follow an object, reading in the distance to the object and calculating the linear and angle speeds using the functions shown in code snippets \ref{code:linSpeedBreitenburg} and \ref{code:angSpeedBreitengburg}.

\vspace{5mm}
\begin{lstlisting}[caption={Functions for calculating the angular speed of the E-Puck},label=code:angSpeedBreitengburg, float]
int angle_speed_calc(int pos, int gain)
{
	int consigne = 0;
	int angle_speed = 0;
	int ecart = consigne - pos;

	angle_speed = ecart*gain;

	if(angle_speed >= 1000)
		angle_speed = 999;
	else if(angle_speed <= -1000)
		angle_speed = -999;

	return angle_speed;
}
\end{lstlisting}

\begin{lstlisting}[caption={Functions for calculating the linear speed of the E-Puck},label=code:linSpeedBreitenburg, float]
int lin_speed_calc(int distance, int gain)
{
	int consigne = 100;
	float h = 0.05;
	int ti = 3;
	int ecart = consigne-distance;
	int lin_speed;

	ui_lin = ui_lin + h * ecart / ti;
	lin_speed = (ecart + ui_lin) * gain;

	if(lin_speed >= 1000)
	{
		ui_lin = 999/gain - ecart;
		if(ui_lin > 60)	
			ui_lin = 60.0;
		lin_speed = 999;
	}
	else if(lin_speed <= -1000)
	{
		ui_lin = -999/gain + ecart;
		if(ui_lin < -10)
			ui_lin = -10.0;
		lin_speed = -999;
	}
	return lin_speed;
}
\end{lstlisting}

The angular speed is calculated by passing the function the difference in the readings between the 2 front left infrared sensors and the 2 front right infrared sensors. The magnitude of this value then decides how quickly the robot must turn left or right, whilst whether the value is positive or negative decides which way the robot should turn. 

The linear speed calculator works in a similar way where the average of the front 2 middle infrared sensors are passed into the function and dependent upon the magnitude of this value the linear speed can then be calculated. 

These values are then added together for the speed of the right wheel and the angular speed is subtracted from the linear speed for the left wheel causing the E-Puck to move in the desired way.

\subsection{Following Light}

The `finding\_light' function finds light using the infrared sensors once again, although instead of detected the emitted infrared light the sensors detect the ambient light.

One anticipated obstacle for this function would be for different environments the E-Puck would behave very differently. For example, in very dark environments the E-Puck will find it very easy to find a light source. Although, in brighter environments the E-Puck will find it much more difficult to find the light source being shone onto it. To attempt to overcome this problem a baseline reading of the environment is taken before the E-Puck begins to look for any light source. From here the E-Puck can compare any future readings with the baseline readings and thus find any increases in light, being any newly introduced light sources.

\subsection{Avoiding Light}

The `avoid\_light' function works in a very similar way to the `finding\_light' function although instead of being attracted to the light source it moves in the opposite direction.

\subsection{Breitenburg Shocker}

Once again, the `run\_breitenberg\_shocker' function runs in a very similar way to the `run\_breitenberg\_follower' but instead of moving towards an objected when one is discovered on the front infrared sensors the E-Puck turns and moves away from it.

\subsection{Follow Hand}

As one may realise the `followhand' function is more complex and therefore contains more problems to solve. The idea behind the function is to check whether there is object nearby the front 2 infrared sensors, if so the camera takes an image and checks whether or no the image is a hand. The camera does this by checking the colour of the image and recognising whether it is the range of the hue of a hand.

The first stage of this function is very simple and can be done easily, in the same way as previously seen functions by just reading the distance from an infrared sensor on the E-Puck.

These sensors then trigger whether an image should be taken. If the image is to be taken then it is done so by calling the `hgetImage' function which has been taken from the aforementioned demo project and renamed. The following `hImage' function then begins to process the image, firstly converting the RGB image to HSV. The code and algorithm for this can be seen in code snippet \ref{code:RGB2HSV}. One may notice that only the hue has been calculated and this is because only the hue is needed as the saturation and value intensity are both environment dependent. Therefore finding a hand in a brightly light room or a poorly light room should obtain similar results.

\vspace{5mm}
\begin{lstlisting}[caption={Converting RGB pixels to HSV},label=code:RGB2HSV,float]
//RGB turned into an integer value for comparison
red = (hbuffer[2*i] & 0xF8);
green = (((hbuffer[2*i] & 0x07) << 5) | ((hbuffer[2*i+1] & 0xE0) >> 3));
blue = ((hbuffer[2*i+1] & 0x1F) << 3);
// convert to RGB to Hue
double hue = RGB2Hue(red, green, blue);
if(hue> 6 && hue<38){
	hnumbuffer[i] = 1;
	vis +=1;
}else{
	hnumbuffer[i] = 0;
}
\end{lstlisting}

Once the hue has been calculated the function proceeds to categorise each pixel as a 1 if the hue lies within the range of a human hand or a 0 otherwise. The range the hue must lie within starts at 6\textsuperscript{o} through to 38\textsuperscript{o}. These values have been calculated after careful research was carried out on human skin colour, referred to in section \ref{sec:background}.

The final stage of the function is to check the amount of times this hue has been seen and calculate a threshold where the E-Puck decides whether human skin is visible or not. This threshold was discovered after testing various threshold values which will be analysed in more detail throughout section \ref{sec:ResultsTesting}. 
