\section{Conclusion}
The conclusion will begin by discussing any improvements that could be made to the projects or anything that the group would change whilst completing the project. It will follow by stating what was done well by the group and what could have been improved to achieve a better result.
\subsection{Improvements}
Whilst the group worked to the best of their ability to produce the work described throughout this report there are a couple of improvements that the group would have liked to implement given more time. One of these improvements would be to correct the issue described towards the end of section \ref{sec:Results}. The issue was that the second E-Puck struggled to follow the first E-puck consistently.

To attempt to reduce the effect of turning the first E-Puck too quickly and the second E-Puck losing track of it, one may wish to incorporate more than just the front 4 infrared sensors on the second E-Puck. This would allow the second E-Puck to detect if the first E-Puck has turned significantly enough to pass the side of the second E-Puck and thus allowing better detection. Whilst this seems relatively simple, because of the complex formula used to move the E-Puck in the demo `run\_breitenburg\_follower' function, it becomes much more complicated to achieve the ideal situation.

One other improvement the group could have made collectively throughout the entire project was to ensure better communication was established between each group member to allow everybody to know what was being developed at each stage. This would lead to faster cooperation on tasks and a greater understanding by all.

Overall this project has been a success in achieving what it had been set out to achieve, with each group member contributing towards the end target goal. Each task has been successfully solved with only small errors being produced in some of the multiple solutions.